\documentclass{SYSUDaily}

\usepackage{listings}
\usepackage{xparse}
\usepackage{fbox}
\usepackage{amsmath,amssymb,amsfonts}

\setmainfont{MinionPro}[
  Extension = .otf,
  Ligatures={TeX, Common},
  Numbers={Monospaced,Proportional},
  UprightFont = *-regular,
  ItalicFont = *-It,
  BoldFont = *-bold,
  Kerning=On,
  SizeFeatures = {
    {Size = -9, Font = *-capt},
    {Size = 9-13.01, Font = *-regular},
    {Size = 13.01-19.91, Font = *-subh},
    {Size = 19.91-, Font = *-disp}
    },
  SlantedFeatures = {Font = *-regular, FakeSlant = .15},
  BoldSlantedFeatures = {*-bold, FakeSlant = .15}
]
\setmathfont{MinionMath-Regular.otf}[
  SizeFeatures = {
    {Size = -6, Font = MinionMath-Tiny.otf,
        Style = MathScriptScript},
    {Size = 6-8.4, Font = MinionMath-Capt.otf,
        Style = MathScript},
    {Size = 8.4-13, Font = MinionMath-Regular,
        Style = MathScript},
    {Size = 13-, Font = MinionMath-Subh.otf},
    },
]
\setmathfont[version = bold]{MinionMath-Bold.otf}[
  SizeFeatures = {
      {Size = -6, Font = MinionMath-Bold.otf,
          Style = MathScriptScript},
      {Size = 6-8.4, Font = MinionMath-Bold.otf,
          Style = MathScript},
      {Size = 8.4-13, Font = MinionMath-Bold,
          Style = MathScript},
      {Size = 13-, Font = MinionMath-BoldSubh.otf},
    },
]

\DeclareMathAlphabet{\panscr}{U}{rsfs}{m}{n}
\DeclareMathAlphabet{\mycal}{OMS}{cmsy}{m}{n}
\DeclareMathAlphabet{\mybcal}{OMS}{cmsy}{b}{n}
% \DeclareMathAlphabet{\symcal}{OMS}{cmsy}{m}{n}
% \DeclareSymbolFont{cmsy}{OMS}{cmsy}{m}{n}
% \DeclareSymbolFontAlphabet{\symcal}{cmsy}
\setmathfont[range = {tt}, Scale = MatchUppercase ]{Garamond-Math.otf}
\setmathfont[range = sfup]{Garamond-Math.otf}
% \setmathfont[range={up/Greek,"2211, "220F, cal, bfcal}]{Neo Euler.otf}
\setmathfont[range={bfsfit/{greek, Greek},frak,scr}]{Garamond-Math.otf}
\setmathfont[range={"222B,"222C,"222D},StylisticSet ={7,9}]{Garamond-Math.otf}
\setmathfont[range={"2192,"21A6,"21AA,"2972}]{NewComputerModernMath}
\setmathfont[range={}]{MinionMath-Regular.otf}
% \setmathfont[range={"2B47"}]{NewCM10-Book.otf}
\renewcommand\.{\mathchoice{\mkern2mu}{\mkern2mu}{\mkern1mu}{\mkern1mu}}

\IfFontExistsTF{方正新书宋.TTF}{
  \setCJKmainfont{方正新书宋.TTF}[
  ItalicFont = STKAITI.TTF,
  BoldFont = 方正粗雅宋_GBK,
]
}{}
\IfFontExistsTF{苹方黑体-细-简.ttf}{
  \setCJKsansfont{苹方黑体-细-简.ttf}[
    BoldFont = 苹方黑体-细-简.ttf
  ]
}{}
\IfFontExistsTF{苹方黑体-细-简.ttf}{
  \setsansfont{苹方黑体-细-简.ttf}[
    BoldFont = 苹方黑体-细-简.ttf
  ]  
}{}

\lstset{
	basicstyle=\small\ttfamily,
	keywordstyle=\color{black}\bfseries\underbar,
	identifierstyle=,
	stringstyle=\ttfamily,
	showstringspaces=false
	}
\begin{document}

本项目是中山大学数学学院每日一题的模板, 使用方法为:
\begin{lstlisting}
\begin{daily}[类型, 日期, 星级, 标题]
	正文
\end{daily}
\end{lstlisting}
方括号\texttt{[]}中的参数的顺序不影响结果. 其中类型如下:
\begin{quote}
	\makeatletter
	\newcommand\TestColour[1]{{\fboxsep0pt\fbox{\colorbox{#1}{\phantom{XX}}}}}
	\makeatother
	\linespread{1}\selectfont
	\ExplSyntaxOn
	\begin{itemize}
		\clist_map_inline:Nn \c_math_styles_clist {
		\item[\TestColour{#1color}] \texttt{#1}
			}
	\end{itemize}
	\ExplSyntaxOff
\end{quote}

日期未设置时, 默认为当天日期, 未设置类型时, 默认为命题:
\begin{daily}
	今天是\today{}.
\end{daily}

\begin{daily}[abc, theorem, 11111111, 3]
	令$\mathbb I$是$\mathbb R$中的某开区间,令$f\in C^\infty(\mathbb  I)$,称其在$x_0$处解析,若在$x_0$某邻域内下式成立:
	$$
	f(x) = \sum_{n\geqslant0} a_n(x-x_0)^n.
	$$
	记为$f\in C^\omega(x_0)$。任给$E\subset \mathbb I$,定义$C^\omega(E)=\bigcap_{x\in E}C^\omega(x)$。

	现在证明:
	\begin{enumerate}
		\item 倘若 $$ \sup_{n\geqslant 1} \frac1n \log\Bigl(\frac{\sup_{x\in \mathbb I}|f^{(n)}(x)|}{n!}\Bigr)<\infty. $$ 那么$f\in C^\omega(\mathbb I)$。
		\item 倘若 $f\in C^\omega(x_0)$,那么在$x_0$的某邻域$J$内,下式成立:$$ \sup_{n\geqslant 1} \frac1n \log\Bigl(\frac{\sup_{x\in J}|f^{(n)}(x)|}{n!}\Bigr)<\infty. $$
		\item 证明以下解析函数的刻画:
			\begin{enumerate}
				\item 第二款的逆命题;
				\item $f\in C^\omega(\mathbb I)$当且仅当任给$[\alpha,\beta]\in \mathbb I$,均有 $$ \sup_{n\geqslant 1} \frac1n \log\Bigl(\frac{\sup_{\alpha<x<\beta}|f^{(n)}(x)|}{n!}\Bigr)<\infty. $$
			\end{enumerate}
	\end{enumerate}
\end{daily}


% \arabic{year}\ifnum\month<10 0\arabic{month}
% \else\arabic{month}
% \fi
% \if\arabic{day}<10	0\arabic{day}
% \else\arabic{day}
% \fi
% \number\year\ifnum\month<10 0\number\month\else\number\month\fi\ifnum\day<10 0\number\day\else\number\day\fi
% \scalebox{.3}{\makebox[1ex]{\tikz{\node[star, star point ratio=2.5, draw, fill=red]at (1,1) {};}}} wrtqbjzxbnkgb h
\end{document}
