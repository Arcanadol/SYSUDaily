\documentclass{SYSUDaily}

\usepackage{listings}
\usepackage{fbox}
\usepackage{multicol}
\usepackage{cleveref}
\RequirePackage[a4paper, centering]{geometry}



\lstset{
	basicstyle=\small\ttfamily,
	keywordstyle=\color{black}\bfseries\underbar,
	stringstyle=\ttfamily,
	showstringspaces=false,
  basewidth=.525em,
  tabsize=4,
	}
\ctexset{
  section/format += \raggedright,
}
\title{中山大学数学学院「每日一题」模板}
\author{Innocent, Panadol, \emph{v2.1}}

\makeatletter
\renewenvironment{abstract}[1][\abstractname]{
  \small
  \begin{center}
    {\bfseries #1\vspace {-.5em}\vspace {\z@ }}
  \end {center}
  \begin{quotation}
}{\end{quotation}}
\makeatother
\begin{document}
\maketitle

\begin{abstract}[简介]
	此文档类 \verb|SYSUdaily.cls| 旨在建立一个简单易用的中山大学数学学院「每日一题」计划的排版渠道。
\end{abstract}

\begin{abstract}[免责声明]
	此文档类仅仅为了处理中山大学数学学院「每日一题」计划而编写,勿作他图,任何将此模板用于非中山大学数学学院「每日一题」计划排版而导致的纠纷与本模板作者无关。
\end{abstract}

\begin{multicols}{2}
	\tableofcontents
\end{multicols}


\section{介绍}
最早的一版用于中山大学数学学院「每日一题」计划排版的 \LaTeX{} 模板由 Innocent 编写,在 2023 年 10 月投入最初的使用。2024年3月,Panadol重构了整个模板的代码,现在代码托管在GitHub\footnote{\url{https://github.com/Arcanadol/SYSUdaily}}上,由Innocent和Panadol共同维护。

\section{使用方法}
\subsection{\texttt{daily}环境}
\texttt{daily}环境是本模板主要实现的功能,使用方法为:
\begin{lstlisting}
  \begin{daily}[参数]
    正文
  \end{daily}
\end{lstlisting}
方括号 \verb|[]| 中的参数的顺序不影响结果。参数列表如下:

\paragraph{类型}
仅需输入类型命名即可确定 \verb|SYSUdaily| 的类型。所有类型定义及颜色如下:
\begin{quote}
	\makeatletter
	\newcommand\TestColour[1]{{\fboxsep0pt\fbox{\colorbox{#1}{\phantom{XX}}}}}
	\makeatother
	\linespread{1}\selectfont
	\ExplSyntaxOn
	\begin{multicols}{2}
		\begin{itemize}
			\seq_map_inline:Nn \c_math_styles_seq {
			\item[\TestColour{#1color}]
				\texttt{#1}
				}
		\end{itemize}
	\end{multicols}
	\ExplSyntaxOff
\end{quote}
类型未设置时,默认为 \verb|proposition|。
\paragraph{日期}
日期为8、6、4位整数:\verb|20101225|、\verb|101225|、\verb|1225| 分别代表三种不同的日期输入风格。

\begin{multicols}{2}
	% \footnotesize
	\begin{lstlisting}
  \begin{daily}[19111225]
    那是一个遥远的圣诞夜。
  \end{daily}
  \end{lstlisting}
	\small
	\begin{daily}[19111225]
		那是一个遥远的圣诞夜。
	\end{daily}
\end{multicols}


\begin{multicols}{2}
	% \footnotesize
	\begin{lstlisting}
  \begin{daily}[111225]
    我忘不了那年的圣诞节。
  \end{daily}
  \end{lstlisting}
	\small
	\begin{daily}[111225]
		我忘不了那年的圣诞节。
	\end{daily}
\end{multicols}


\begin{multicols}{2}
	% \footnotesize
	\begin{lstlisting}
  \begin{daily}[1225]
    今年的圣诞节,还要多久?
  \end{daily}
  \end{lstlisting}
	\small
	\begin{daily}[1225]
		今年的圣诞节,还要多久?
	\end{daily}
\end{multicols}

一个会被识别为日期的数字,但是格式错误时,将不会显示任何东西:
\begin{multicols}{2}
	% \footnotesize
	\begin{lstlisting}
  \begin{daily}[125]
		今天是 1 月 25 日?异或 12 月 5 日?
  \end{daily}
  \end{lstlisting}
	\small
	\begin{daily}[125]
		今天是 1 月 25 日?异或 12 月 5 日?
	\end{daily}
\end{multicols}
请不要把它当做关闭日期显示的开关,我们提供 \verb|\SYSUDailyNoShowDate| 来关闭日期显示,也可以用 \verb|\SYSUDailyDoShowDate| 来恢复开启。
\begin{multicols}{2}
	% \footnotesize
	\begin{lstlisting}
	\SYSUDailyNoShowDate
	\begin{daily}
		今天几月几号,我也不知道。
	\end{daily}
  \end{lstlisting}
	\small
	\SYSUDailyNoShowDate
	\begin{daily}
		今天几月几号,我也不知道。
	\end{daily}
\end{multicols}
\SYSUDailyDoShowDate
日期未设置时,默认为当天日期 \verb|\today|:
\begin{multicols}{2}
	% \footnotesize
	\begin{lstlisting}
  \begin{daily}
    今天是\today{},不是么?
  \end{daily}
  \end{lstlisting}
	\small
	\begin{daily}
		今天是\today{},不是么?
	\end{daily}
\end{multicols}



\paragraph{难度}
难度为1--4的整数,否则不会识别为难度。
\begin{lstlisting}
  \begin{daily}[解析函数的刻画, theorem, 20101010, 4]
    ……
  \end{daily}
\end{lstlisting}
\begin{daily}[解析函数的刻画, theorem, 20101010, 4]
	令$\mathbb I$是$\mathbb R$中的某开区间,令$f\in \mathscr{C}^\infty(\mathbb  I)$,称其在$x_0$处解析,若在$x_0$某邻域内下式成立:
	\[
	f(x) = \sum_{n\geqslant0} a_n(x-x_0)^n.
	\]
	记为$f\in \mathscr{C}^\omega(x_0)$。任给$E\subset \mathbb I$,定义$\mathscr{C}^\omega(E)=\bigcap_{x\in E}\mathscr{C}^\omega(x)$。

	现在证明:
	$f\in \mathscr{C}^\omega(\mathbb I)$当且仅当任给$[\alpha,\beta]\in \mathbb I$,均有
	\[
			\sup\biggl\{~\biggl| \frac{f^{(n)}(x)}{n!} \biggr|^{1/n}\!;~n\geqslant 1, x\in [\alpha,\beta] ~\biggr\}   <\infty.
	\]
\end{daily}
难度未设置时,默认为 \verb|0|。
\paragraph{标题}
除去满足上面条件的参数,其余参数均视为标题,若识别到多个标题,则以第一个为准。
若想要输入域时间,难度,类型重复的标题,可以在使用 \verb|\mbox{}| 解决:
\begin{multicols}{2}
	\begin{lstlisting}
    \begin{daily}[\mbox{theorem}]
      正文
    \end{daily}
  \end{lstlisting}
	\small
	\begin{daily}[\mbox{theorem}]
		正文
	\end{daily}
\end{multicols}
标题未设置时,默认为 \verb|\relax|。

\subsection{字体}
\verb|SYSUdaily.cls| 默认使用 \textsf{fandol} 字库,因此不支持 pdf\LaTeX{},同时,
在检测到存在时,会调用Resource Han Rounded作为中文等宽字体。

当使用 \verb|unimath=true| 预设时,会使用方正新书宋、方正粗雅宋、方正新楷体、Noto Sans CJK SC作为中文字体,且使用 Minion Pro 作为西文字体,除去Noto Sans CJK SC外,以上字体均是商业字体。

\subsection{数学字体}
\label{ssec:math_fonts}
\verb|SYSUdaily.cls| 默认使用以下宏包作为字体调用库:
\begin{itemize}[parsep=0pt,itemsep=0pt]
	\item
			\textsf{amsfonts}
	\item
			\textsf{amssymb}
	\item
			\textsf{mathrsfs}
	\item
			\textsf{dsfont}
	\item
			\textsf{eucal}
\end{itemize}
当使用 \verb|unimath=true| 预设时,会使用 Minion Math 作为主要数学字体,这款字体也是商业字体。

\section{宏包依赖}

在任何情况下,文档类都会显式调用以下宏包或文档类:
\begin{itemize}[parsep=0pt,itemsep=0pt]
	\item
			\textsf{ctexart}。中文排版的通用框架;
	\item
			\textsf{xcolor}。提供色彩支持,默认按 \verb|svgnames| 载入;
	\item
			\textsf{amsmath}、\textsf{amsthm}。数学框架;
	\item
			\textsf{fontspec}。调整字体,因此不支持 pdf\LaTeX{};
	\item
			\textsf{geometry}。用于调整页面尺寸;
	\item
			\textsf{eucal}。调整手写数学字体样式;
	\item
			\textsf{fixdif}、\textsf{mathtools}。\textsf{amsmath}的扩展;
	\item
			\textsf{kvoptions}。设置关键词之用;
	\item
			\textsf{graphicx}。提供图形插入的接口;
	\item
			\textsf{enumitem}。设置列表环境格式。
\end{itemize}
\verb|unimath=true| 预设会调用相关的字体宏包,具体调用细节已在第 \ref{ssec:math_fonts} 小节中列出。

\section{更新历史}
\begin{flushleft}
	\footnotesize
	\textbf{v2.1}~(2024/04/04--2024/04/10)
	\begin{itemize}[parsep=0pt,itemsep=0pt]
		\item
				更精确的日期检测(大于100的4、6、8 位整数),添加了开启和关闭日期显示的功能。
	\end{itemize}
  
	\textbf{v2.0}~(2024/03/20--2024/04/04)
	\begin{itemize}[parsep=0pt,itemsep=0pt]
		\item
				应用\LaTeX{3}重构代码;
		\item
				添加 \verb|unimath| 文档类选项;
		\item
				将原来的 \verb|theorem| 等环境简化为 \verb|daily| 环境;
		\item
				使用 \textsf{l3seq} 处理 \verb|daily| 环境的可选参数,利用更合理的判断机制实现输入上的简化;
		\item
				去除随机图标。
	\end{itemize}
\end{flushleft}

\end{document}
