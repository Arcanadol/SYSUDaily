\documentclass{daily}

\renewcommand{\epigraphsize}{\footnotesize}

\begin{document}
\begin{center}
    \Large\sffamily 从Baire函数分类到Borel可测函数
\end{center}

\section*{引\kern\ccwd 子}

\begin{epigraphs}
	\qitem{小白兔戴上了眼镜。“请问陛下,我该从何说起呢?”他问道。“从头开始,”国王严肃地说,“一直说到最后,然后停下来。”}{路易斯·卡罗尔}
\end{epigraphs}

我们从Riemann积分时期开始回顾。在19世纪80年代,数学家们已经知道由很多(至少是当时这么认为)病态函数,最出名一个的是Dirichlet函数:
\[
	\mathbb 1_{\mathbb Q}(x) \coloneqq
	\begin{dcases}
		1, & x\in\mathbb Q;    \\
		0, & x\notin\mathbb Q.
	\end{dcases}
\]

其巨大的不连续性是病态的来源,在当时已经知道Dirichlet函数不是一个可积函数。因此,在当时有人希望对函数的不连续行进行分类,从而区分连续可微的,Riemann可积的,Riemann不可积或者性质更差的函数。Hermann Hankel在1970年提出了以下分类方式:
\begin{enumerate}
	\item 若$f\colon [a,b]\to\mathbb R$连续,则称其为第$1$类。在以往的分析学对象中,这类函数是最重要的;
	\item 若函数$f\colon[a,b]\to\mathbb R$只有有限个不连续点,则归为第$2$类;
	\item 若函数$f\colon[a,b]\to\mathbb R$有无穷个不连续点,但在$[a,b]$的某一个稠密集上连续,则归为第$3$类,反之归为第$4$类。
\end{enumerate}
Dirichlet函数明显属于第$4$类,而Thomae函数:
\[
	T(x) \coloneqq
	\begin{dcases}
		1/q, & \textit{$x = p /q\in\mathbb Q$且$p$, $q$互素;} \\
		0,   & x\notin\mathbb Q.
	\end{dcases}
\]
是Riemann可积的,其属于第$3$类(在无理数上连续)。因此Hankel在考察了当时的例子后得出了\textit{「Riemann可积当且仅当函数是$1$、$2$、$3$类的」}的结论。很遗憾,这是错误的,并且反例颇为精妙。虽然Hankel失败了,但是对函数分类的想法仍会一致持续下去……

\section{\texorpdfstring{\simpleruby{勒内}{René}-\simpleruby{路易}{Louis}·\simpleruby{贝尔}{Baire}}{René-Louis Baire}}

\begin{epigraphs}
	\qitem{集连续函数谓之类零,而由连续函数极限所成之不连续函数,谓之类一。}{勒内-路易·贝尔}
\end{epigraphs}

在Hankel之后,René-Louis Baire是另一位在19世纪对函数实行分类的数学家。与被束之高阁的Hankel分类不同,Baire对函数的分类具有远超当时时代的重要价值,这种价值直到Lebesgue理论和集合论的完善才逐渐显现出它的价值。

正如引言所述,Baire的函数分类可以用数学的语言如下描述:
\[
	\mathfrak B_0(\mathbb R) = C(\mathbb R),\quad \mathfrak B_1(\mathbb R) = \set{f\colon \mathbb R\to\mathbb R \given \textit{存在$\{f_n\}_{n\geqslant 0}\subseteq  \mathfrak B_0(\mathbb R)$, $f_n \to f$}}\setminus \mathfrak B_0(\mathbb R).
\]

\begin{daily}{proposition}{20240329}{可微函数}
	证明对任意可微函数$f\colon \mathbb R\to\mathbb R$, $f'$均是$\mathfrak B_1$的。
\end{daily}

因此,$\mathfrak B_1$函数可能是不连续的,但可以自然地认为其「不良性」应该不会太差: 事实上,按照Hankel的分类,$\mathfrak B_1$函数均是Hankel第$3$类的。Baire的高类函数:$\mathfrak B_n$定义如下:
\[
	\mathfrak B_n(\mathbb R) = \set[\bigg]{f\colon \mathbb R\to\mathbb R \given \textit{存在$\{f_n\}_{n\geqslant 0}\subseteq  \bigcup_{k=0}^{n-1}\mathfrak B_k(\mathbb R)$, $f_n \to f$}}\setminus \bigcup_{k=0}^{n-1}\mathfrak B_k(\mathbb R).
\]



也就是可以用低类函数序列逐点逼近的全体函数。可以想象,这样的函数的性质肯定会越来越差,直到$\mathfrak B_n(\mathbb R)$成为空集,亦或是永不停歇。

得到定义之后,我们先来考虑在当时十分出名的函数的分类:

\begin{daily}{proposition}{20240330}{Dirichlet函数}
	引发Riemann可积性危机的Dirichlet函数$\mathbb 1_{\mathbb Q}$是$\mathfrak B_2$或者$\mathfrak B_1$的。
\end{daily}

事实上,在$20$世纪之前,数学家们处理的函数大多都是低类函数,\textit{「任意高类的函数都存在」}这个命题在当时估计不会有解答。Baire本人也也因为疾病早早地结束了学术生涯,这些分类的理论最终由Lebesgue、Hausdorff、Cantor等人的理论解答……同时,新的数学也在酝酿之中。

下面的问题难度会上升,但是不做题不影响文字材料阅读。

\section{从函数到集合}

\begin{epigraphs}
	\qitem{集论乃其中关键;简言之,于吾等所立之见地,凡函数之问,皆牵至集论之疑;唯当此疑得以深探前瞻,始能略解所提之难。}{勒内-路易·贝尔}
\end{epigraphs}

我们有理由相信,研究函数的问题最终会变成研究集合的问题:且先不论函数就是一个集合,单单研究函数如何描述集合的对应就可以很大程度上决定函数的性质了:

\begin{daily}{definition}{20240330}{原像}
	令$f\colon X\to Y$, $Z\subseteq  Y$,则定义:
	\[f^{-1}(Z) = \set{x\in X\given f(x)\in Z},\quad f^{-1}(a) = f^{-1}(\{a\}).\]
\end{daily}


\begin{daily}{example}{20240330}{连续函数}
	这是一个朴素的例子:如果$f\colon\mathbb R\to\mathbb R$是连续的,\CJKunderdot{当且仅当}对于任意开集$U\subseteq \mathbb R$,都有:$f^{-1}(U)$是$\mathbb R$中开集。
\end{daily}

\begin{daily}{proposition}{20240331}{序列极限的表示}
	如果$\{f_n\}_{n\geqslant 0}$是$\mathbb R\to\mathbb R$的函数列,且$f_n\to f$。则给定开集$(a,b)\subseteq \mathbb R$,用$f_n^{-1}((a,b))$描述$f^{-1}((a,b))$。同理,用$f_n^{-1}(0)$描述$f^{-1}(0)$。这道题可能需要一点「描述」的技巧(我们可能会在实变函数中学到一些)。

	(提示:你可以写出$f_n\to f$的命题表达,然后考虑$\forall$、$\exists$这些量词能否对应$\bigcup$或$\bigcap$)
\end{daily}

结合以上的命题我们发现,对于第$\mathfrak B_1$函数,我们可以用一些可数交并的交错排列来生成我们想要的「$\mathfrak B_1$原像」。为了更方便地说明这件事,我们引入一些看起来非常吓人的记号。需要提前说明的是,为了方便,这些记号经过一些修改,\emph{与惯用的记号不符}:惯用的记号会用粗体,且有$0$作为上标。

\begin{daily}{definition}{20240331}{有限Borel分层}
	我们只在$\mathbb R$中讨论。将$\mathbb R$中所有开集记为$\Sigma_1$,闭集记为$\Pi_1$。\CJKunderdot{归纳}定义如下:
	\[\Sigma_n\coloneqq\set[\bigg][~]{\bigcup_{k\geqslant 0}A_k\given \forall k\geqslant 0, A_k\in \Pi_{n-1}},\quad \Pi_n=\set{A^\complement\given A\in\Sigma_n}.\]
	看上去这些记号有点吓人,但是我们如果算一些例子的话:
	\begin{itemize}
		\item $\Sigma_2$是$\Pi_1$,也就是闭集的可数并,我们可以认为是$F_{\sigma}$集($F$是德语「闭集」~Fermé的缩写,$\sigma$是法语「并」~Somme的转写)。
		\item 相应地,$\Pi_2$是开集的可数交。
	\end{itemize}
\end{daily}

下面这道题是想要说明我们的Borel分层是上升的,显而易见,阶越大的集合越复杂。将此题归类到$\bigstar\,\bigstar$是因为有新定义,其在无意中拔高了难度,实际上难度并不高。
\begin{daily}{proposition}[Borel分层·上升]
	证明$\Sigma_1\subseteq  \Sigma_2\subseteq  \Sigma_3\subseteq \cdots$, $\Pi_1\subseteq  \Pi_2\subseteq  \Pi_3\subseteq \cdots$。更一般地,$\Sigma _n\cup \Pi _n\subseteq  \Sigma _{n+1}\cap \Pi _{n+1}$。
	\hint{}
\end{proposition}


Borel分层和Baire函数分类并不是同一个人实行的,但他们之间有千丝万缕的联系:以下是对引言中「函数论的问题$\to$集合论的问题」的一个佐证:
\begin{proposition}[2]{0}[Borel分层和Baire函数分类的联系:$\mathbb R$情形]
	证明对$\mathfrak B_n$类函数,均有:
	\[
		f\in\mathfrak B_n \implies \textit{任意开集\(U\subseteq \mathbb R\),均有\(f^{-1}(U)\in \Sigma_{n+1}\)}.
	\]
	\hint{用可数个闭集的并得到开集,然后归纳。}

	注:事实上,反之也对,不过证明是不平凡的。更具体的证明可以见[Alex95]\textsuperscript{(24.3, 24.10)}。
\end{proposition}

\section{Borel可测函数}

\epigraph{数学分析中有一个类似于“先有鸡还是先有蛋”的难题:Lebesgue积分和Lebesgue测度哪个在前面?于我而言都不是;首先出现的是$\mathfrak L^1$空间。}{拉克斯·彼得}




由命题8,我们暂且可以把处理高类函数和处理$\Sigma_n$等同起来。我们现在回到那个时代最具有挑战性的分析问题之一:
\begin{center}
	\itshape Riemann可积的\CJKunderdot{充要条件}是什么?如何推广Riemann积分?
\end{center}
Baire的函数分类没能解决这个问题:Riemann可积函数$\mathfrak R([a,b])$并不能说是在哪一类函数里。事实上,存在可微函数有非Riemann可积的导数,也就是$\mathfrak B_1\setminus\mathfrak R\neq \varnothing$。如果我们想要推广Riemann积分的话,Baire的函数分类是一个良好的平台。

把Riemann积分推广需要几步? 答案是$3$步:$(1)$.~推广可积函数;$(2)$.~推广积分;$(3)$.~完成推广。我们先结合之前学到的Baire函数分类和Borel层级,从第$(1)$步开始看起。事实上,我们在这里只考虑第$(1)$步,具体的计算我们会交给实变函数,但值得一提的是,在Lebesgue本人构建积分的那个时代,这些思想也是不平凡的。

我们称$\bigcup_{n\geqslant 0}\mathfrak B_n$是「可构造的函数」(只是一个临时的术语)。也就是可以写成:
\[
	f=\lim_{i_n\to\infty}\lim_{i_{n-1}\to\infty}\cdots\lim_{i_1\to\infty} f_{i_1\,[-1] ,\dots,\,[1] i_n},\quad n\in\mathbb N_{\geqslant 0},\quad f_{i_1\,[-1] ,\dots,\,[1] i_n}\in C(\mathbb R).
\]
的所有函数$f$。注意,这些极限是不可交换的。

上式看起来非常令人生畏,但在对$\mathbb R$还没有全面认知的$19$世纪,这些函数可能并不会比在稠密集上肆意变动的$\mathbb 1_{\mathbb Q}$更糟糕。那么问题是:
\begin{center}
	\itshape 若$\{f_n\}_{n\geqslant 0}$是一列可构造函数,那么$\lim\limits_{n\to \infty}f_n$ (如果存在)是可构造的吗?
\end{center}
可能任何一位先于Cantor的数学家都会说「是的,这就是Riemann可积函数的推广」,这无非是在一堆极限后面再增加一个罢了。但如果$f_n\in\mathfrak B_n$,也就是极限会不断地增多,那么再添加一个极限这个操作显得不那么直观。换言之,假定$F\colon\mathfrak B^n\mapsto \mathfrak B^{n+1}$是取序列极限并将它们并入集合的操作,那么
\[
	\bigcup_{n\geqslant 0}\mathfrak B_n = \bigcup_{n\geqslant 0}F^n( \mathfrak B_0 ).
\]
Cantor引人注目的发现(当然他当时处理的不是这个问题,但是是类似的)是$F\bigl(\bigcup_{n\geqslant 0}F^n( \mathfrak B_0 )\bigr)$和$\bigcup_{n\geqslant 0}F^n( \mathfrak B_0 )$是可能不一样的。这也意味着自然数之上还有更大的「用来归纳的东西」,在后来我们称其为序数。在这里我们没办法讨论关于序数的细节,详情可以见任意一本基础的集合论教材。

如果读者在$19$世纪发现这件事情的话,此发现将震撼整个分析界:原来即使是构造一个关于序列极限封闭的函数族也是那么困难的事情,原来$\mathbb R$中的开集居然可以允许那么多次可数操作而不稳定下来。事实上,如果借用Hausdorff的拓扑语言,我们可以认为这是一个(非构造性的)「闭包」,这个闭包有个大名鼎鼎的名字:Borel可测函数。

如果只是要知晓闭包的直观,拓扑的语言不是必须的,我们在下一个命题中阐述了我们所要求的闭包:

\begin{definition}{0}[序列极限封闭函数族]
	令$\mathfrak F$是一族$\mathbb R\to\mathbb R$的函数,倘若$\{f_n\}_{n\geqslant 0}\subseteq \mathfrak F$可以得到$\lim_{n \to \infty} f_n\in\mathfrak F$,那么就说$\mathfrak F$是一个序列极限封闭函数族。
\end{definition}

\begin{proposition}[1]{0}[Borel可测函数的定义]
	类比若$A\subseteq \mathbb R$,则$\overline A = \bigcap_{\textit{$F$是含$A$闭集}}F$。
	证明序列极限封闭函数族的任意交也是序列极限封闭函数族。给定连续函数$C(\mathbb R)$,定义其生成的Borel可测函数族为:
	\[
		\mathfrak L^0(\mathcal B_{\mathbb R}) \coloneqq \bigcap_{\textit{$\mathcal M$是含连续函数的}\atop\textit{序列极限封闭函数族}}\mathcal M.
	\]
	你可以认为序列极限封闭函数族$\mathcal M$是一些闭集,这样下来$\mathfrak L^0(\mathcal B_{\mathbb R})$必然会被约化到最小的,包含$C(\mathbb R)$的序列极限封闭函数族,也是「闭包」一词的直观。
\end{proposition}

根据命题$8$(我们承认将其推广到序数的版本),我们可以将$\mathfrak F$是序列极限封闭函数族与$\set{f^{-1}(U)\given f\in\mathfrak F, \textit{\itshape$U$是$\mathbb R$中开集}}$是对可数交可数并封闭(当然还有补)看做是一件事情。因此,与$\mathfrak L^0(\mathcal B_{\mathbb R})$对应地,存在一个包含开集的可数交可数并补封闭集族$\mathcal B_{\mathbb R}$。我们将满足这个性质的集族称为σ-代数。同时,如果我们可以「超过自然数地」归纳,那么事实上存在一个序数$\omega_1$满足:
\[
	\textit{$\set{\textit{\itshape$x$是序数}\given x<\xi}$\itshape 是可数的对任意序数$\xi<\omega_1$成立}.
\]
$\omega_1$是称为第一不可数序数,直观上,可数交、并操作都无法超过可数的界限,因此:
\[
	\mathfrak L^0(\mathcal B_{\mathbb R}) = \bigcup_{0\leqslant\xi <\omega_1} F^\xi(\mathfrak B_0).
\]
序数的存在性与归纳的合理性可以一言以蔽之:「集合论公理保证」,而这已经是$20$世纪的事情了。

第二节开头的那句话:「凡函数之问,皆牵至集论之疑……」来自Baire的论文\emph{Sur les fonctions de variables r\'eelles}, 1899. 不严谨地说,随着分析学的愈加发展,$19$世纪末的这句话就愈发显得珍贵,在精彩纷呈的$20$世纪前夕,Baire、Cantor等才华横溢的数学家是这句极富洞察性的箴言的忠实践行者。

\section{尾声}

\begin{epigraphs}
	\qitem{谁也无法将我们从Cantor为我们创造的乐园中驱逐出去。}{大卫·希尔伯特}
\end{epigraphs}


这段不太寻常的介绍到了尾声,以Baire函数起,到Borel可测函数结束。事实上,这种引入方法是相当非常规的,它的好处仅仅在能充分讲述σ-代数的引入动机(当然这和历史不符),而对于测度这种「能算出值来的东西」却没有任何描述。但事实上,这一片领域很快就会独立于传统的分析学:对集合操作的复杂性估计最终将在空间上σ-代数的处理导向了集合论。直到几十年以后,数学家们才发现这套问题最终处理的结果依赖于集合论公理的选取,读者也许能在命题$5$中逻辑表达与集合的描述之间的关联中得见一二。

最后是另一些琐碎的问题:为什么我们真的只会讨论可数交并或者可数的序列呢?难道因为我们不可以把不可数多个$0$加起来得到一个正数吗(比如$[0,1]$是正测度集,但是它是不可数个零测度的单点集合之并)?抛开测度的角度不谈,我们仍然从可测函数的观点看待:

\begin{theorem}[3]{0}[连续函数在逐点收敛意义下的闭包]
	如果我们允许不可数的「序列」,比如$\lim_{t\to 0^+}f_t \to f$这类指标集不可数的情形(事实上,我们会称为「网收敛」或者「滤子收敛」),那么连续函数在逐点收敛意义下的闭包是\CJKunderdot{所有函数},而序列逐点收敛意义下的闭包是Borel可测函数。
\end{theorem}

这意味着考虑所有不可数「序列」的同时也丢掉了所有东西。即使是在$19$世纪,人们也会相信不可能给所有函数的都定义积分。

另一个例子: Cantor集$\mathbb Z_{2}^\omega$是一个在经典描述集合论中相当具有「万有」性质的一个集合(比如说,每个紧致度量空间都可以视为Cantor集的连续像),抛去这点不谈,它也是个交换群。不难想象,Cantor集作为所有$0$-$1$序列的集合一定在密码学,通讯理论中有特殊的应用,事实上,我们可以给Cantor集定义Fourier变换,假定$\mathbb Z_{2}^\omega$上的三角函数是$\widehat{\mathbb Z_{2}^\omega}$:
\[
	\widehat{\mathbb Z_{2}^\omega} = \set[\bigg]{f_{\!\boldsymbol{\alpha}}\colon \mathbb Z_{2}^\omega \to \mathbb T,~\{a_n\}_{n\geqslant 0}\mapsto \prod_{n\in \boldsymbol{\alpha}} (-1)^{a_n}\given \boldsymbol{\alpha} \in \mathbb Z_{\geqslant 0}^{<\omega}}.
\]
请允许我来解释一下记号。$\mathbb T=\set{z\in\mathbb C\given |z|=1}$, $Z_{\geqslant 0}^{<\omega}$是由$Z_{\geqslant 0}$的有限字符串组成的集合。因此,如果假定$\bigoplus_{n\geqslant 0} a_n$是有限个非$1$元素的乘积,那么
\[
	\widehat{\mathbb Z_{2}^\omega} = \set[\bigg]{\bigoplus_{n\geqslant 0} \xi_n\given \xi_n(\{a_m\}_{m\geqslant 0}) = (-1)^{a_n}}.
\]
\begin{proposition}[1]{0}[将Cantor集映射到$[0,1]$]
	令$f\colon \mathbb Z_{2}^\omega\to [0,1]$, $\{a_n\}_{n\geqslant 0}\mapsto \sum_{n\geqslant 0}2^{-(n+1)} a_n $。那么证明:
	去掉$[0,1]$的某些可数个点之后,$f$是一个双射。
\end{proposition}
事实上,如果我们在Cantor集上面构建测度(Haar测度),那么$f$实际上是将Cantor集上的Borel结构嵌入$[0,1]$,同时还保持着可积函数的积分值。同理,Cantor集上的三角函数也变成了$[0,1]$上的容易描述的函数,被称为Rademacher函数,其图像类似三角函数,如下图所示。

现在,请你利用上面这些乱七八糟的阐述估计Cantor集上的正规数组成的集合的复杂度:
\begin{theorem}[2]{0}[Cantor集上的正规数]
	Cantor集上的正规数定义为
	\[
		\mathrm N\coloneqq \set[\Big]{\{a_n\}_{n\geqslant 0}\in\mathbb Z_{2}^\omega\given \lim_{N\to\infty}\Pr\bigl(\text{\itshape $a_n=1$,其中$0\leqslant n\leqslant N$}\bigr) = \frac12}.
	\]
	也即是,其字符的出现是随机的,读者可以联系一下Weyl的等分布原理。证明$f(\mathrm N)$的Borel分层不会超过$3$,其中$f(\mathrm N)$定义见命题$12$。
\end{theorem}




\begin{thebibliography}{{Ama}05}
	\bibitem[{Alex}95]{amann05}
	{Alexander S. Kechris}.
	\newblock {\em Classical Descriptive Set Theory}.
	\newblock Graduate Texts in Mathematics, Springer-$\,[-1]$Verlag. 1995.
\end{thebibliography}

\end{document}