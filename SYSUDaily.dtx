% \iffalse meta-comment
% !TeX program  = XeLaTeX
% !TeX encoding = UTF-8
%
% Copyright (C) 2024--2025 by Innocent <InnocentFIVE@outlook.com>
%
% This work may be distributed and/or modified under the
% conditions of the LaTeX Project Public License, either
% version 1.3c of this license or (at your option) any later
% version. The latest version of this license is in:
%
%   http://www.latex-project.org/lppl.txt
%
% and version 1.3 or later is part of all distributions of
% LaTeX version 2005/12/01 or later.
%
% This work has the LPPL maintenance status `maintained'.
%
% The Current Maintainer of this work is Innocent.
%
% This work consists of the files SYSUDaily.dtx,
%           and the derived files SYSUDaily.cls,
%                                 SYSUDaily.cls,
%                                 SYSUDaily.pdf,
%                             and README.md.
%
%<*internal>
\iffalse
    %</internal>
    %
    %<*readme>
    README
    %</readme>
    %
    %<*internal>
\fi
\begingroup
\def\NameOfLaTeXe{LaTeX2e}
\expandafter\endgroup\ifx\NameOfLaTeXe\fmtname\else
\csname fi\endcsname
%</internal>
%
%<*install>
\input docstrip.tex
\keepsilent
\askforoverwritefalse

\preamble

Copyright (C) 2024--2025 by Innocent <InnocentFIVE@outlook.com>

This work may be distributed and/or modified under the
conditions of the LaTeX Project Public License, either
version 1.3c of this license or (at your option) any later
version. The latest version of this license is in:

http://www.latex-project.org/lppl.txt

and version 1.3 or later is part of all distributions of
LaTeX version 2005/12/01 or later.

This work has the LPPL maintenance status `maintained'.

The Current Maintainer of this work is Innocent.

This work consists of the files SYSUDaily.dtx,
and README.md.

\endpreamble

\generate{
    \file{\jobname.sty}        {\from{\jobname.dtx}{package}}
    %<*internal>
    \file{\jobname.ins}        {\from{\jobname.dtx}{install}}
    %</internal>
    \nopreamble\nopostamble
    \file{README.md}           {\from{\jobname.dtx}{readme}}
}

\obeyspaces

\endbatchfile
%</install>
%
%<*internal>
\fi
%</internal>
%
%<class|package>\NeedsTeXFormat{LaTeX2e}[2020/10/01]
%<*!(driver|install)>
%<!readme>\GetIdInfo $Id: SYSUDaily.dtx 0.3 2025-10-28 12:00:00Z Innocent <InnocentFIVE@outlook.com> $
%<package>  {SYSUDaily}
%<package>\ProvidesExplPackage{\ExplFileName}
%<package>{2025-11-09} {} {A package for `Daily Question' for School of Mathematics, Sun Yat-Sen University}
%</!(driver|install)>
%<*driver>
% \PassOptionsToPackage{showframe}{geometry}


\documentclass{fdudoc}
\usepackage{xpatch}
\usepackage{multicol}
\usepackage{SYSUDaily}
\usepackage{fixdif}
\usepackage{lipsum}

\ExplSyntaxOn

\xpatchcmd \__codedoc_print_macroname:nN
{ \__codedoc_get_hyper_target:xN }
{ \__codedoc_get_hyper_target:eN }
{ } { \patchfailed }
\xpatchcmd \__codedoc_print_macroname:nN
{
    \tl_replace_all:Non \l__codedoc_tmpa_tl
    { \c_catcode_other_space_tl }
    { \fontspec_visible_space: }
    \__codedoc_macroname_prefix:o \l__codedoc_tmpa_tl
    \__codedoc_macroname_suffix:N #2
}
{ \tl_replace_all:NVn \l__codedoc_tmpa_tl
    \c_catcode_other_space_tl
    { \fontspec_visible_space: }
    \__codedoc_print_macroname_aux:on
    \l__codedoc_tmpa_tl
    { \bool_if:NT #2 { \__codedoc_typeset_TF: } } }
{ } { }
\makeatletter
\lstnewenvironment{fsyntax}[1][]{%
    \lstset{style=style@syntax, #1}\vspace{-1.5ex}}{}

\AtBeginEnvironment { fsyntax }
{
    \cs_set:Npn \lparen { \textup { ( } }
    \cs_set:Npn \rparen { \textup { ) } }
    \char_set_catcode_active:N |
    \char_set_catcode_active:N <
    \char_set_active_eq:NN | \orbar
    \char_set_active_eq:NN < \syntaxopt@aux
}
\makeatother
\ExplSyntaxOff





\hypersetup{
    pdftitle  = {中山大学数学学院「每日一题」模板},
    pdfauthor = {Innocent, Panadol}}


\title{\textbf{中山大学数学学院「每日一题」模板}}
\author{Innocent, Panadol}

\begin{document}
\DocInput{\jobname.dtx}
%<!--CODEDOC-->  \PrintChanges
\end{document}
%</driver>
% \fi
% 
% \maketitle
% \begin{abstract}
%   \pkg{SYSUdaily}是一个旨在建立一个简单易用的中山大学数学学院%
% 「每日一题」计划的排版渠道的框架,主要功能由宏包%
% \pkg{SYSUDaily}实现。
% \end{abstract}
% \begin{documentation}
% \section{介绍}
% 最早的一版用于中山大学数学学院「每日一题」计划排版的\LaTeX{}模板由 Innocent 编
% 写,在2023年10月投入最初的使用。2024年3月,Panadol重构了整个模板的代码,现在代
% 码托管在GitHub上,由Innocent和Panadol共同维护。
% 2025年10月,Innocent重构了整个模板的代码。

% \section{安装}
% \pkg{SYSUdaily}如今暂无上传CTAN的计划,因此需要手动安装。
% \begin{enumerate}
%   \item 从Github上下载\cls{SYSUDaily}的%
%     \href{https://github.com/Arcanadol/SYSUDaily}{安装包};
%   \item 利用如下Shell命令提取宏包:
% \begin{shellexample}
%   xetex SYSUDaily.dtx
% \end{shellexample}
%   \item 将得到的\pkg{SYSUDaily.sty}文件复制到\TeX{}发行版的
% 本地TDS根目录中对应的 |tex/latex/SYSUDaily| 文件夹内。
%   \item 执行 |mktexlsr| 命令刷新文件名数据库以完成安装。
% \end{enumerate}

% \section{宏包选项}
%  \begin{function}[added=2025-11-11]{lang}
%   \begin{fsyntax}[emph={[1]lang}]
%     lang = (*<\defaultval{Chinese}|English>*)
%   \end{fsyntax}
%   语言选项,\opt{English}选项下,标题命名和标语会与\opt{Chinese}下有区别。
% \end{function}

% \section{使用方法}
% \subsection{\pkg{daily}环境}
% \pkg{daily}环境是本模板主要实现的功能,使用方法为:
% \begin{latexexample}
%   \begin{daily}[参数]
%     正文
%   \end{daily}
% \end{latexexample}
% 方括号 \verb|[]| 中的参数的顺序不影响结果。参数列表如下:

% \begin{function}{daily}
% \begin{fsyntax}
%     \begin{daily}(*\oarg{可选参数}*)
%         内容
%     \end{daily}
% \end{fsyntax}
% 排布每日一题。\oarg{可选参数}是一个用逗号隔离的参数列表,其中参数的顺序不影响结果。
% 为了使用方便,我们并不适用 |key = value| 型的设置,而是利用参数的内容决定参数类型。
% 以下是可能的参数类型。
% \end{function}
% 
% \begin{function}{环境}
% \begin{fsyntax}
%     = (*<empty|theorem|definition|\defaultval{proposition}|lemma|corollary|fact|example|think|calculate|brain>*)
% \end{fsyntax}
% \end{function}
% 仅需输入环境命名即可确定 |daily| 的环境, 
% 选择如下:
% \ExplSyntaxOn
% \cs_new:Npn \TestColour #1 {
%     { \fboxsep0pt \fbox { \colorbox { #1 } { \phantom { XX } } } }
% }
% \begin{multicols}{2}
%     \begin{itemize}
%         \seq_map_inline:Nn \c__SYSUDaily_math_styles_seq {
%         \item [\TestColour{ __SYSUDaily_#1_color }]
%               \texttt{#1}
%               }
%     \end{itemize}
% \end{multicols}
% \ExplSyntaxOff
% 下面是一个实例。
% \begin{multicols}{2}
% \begin{latexexample}
%   \begin{daily}[theorem]
%       这是一个定理。
%   \end{daily}
% \end{latexexample}
%     \small
%   \begin{daily}[theorem]
%       这是一个定理。
%   \end{daily}
% \end{multicols}
% 
% \begin{function}{日期}
% \begin{fsyntax}
%     = (*\defaultval{\cs{today}}*)
% \end{fsyntax}
% 日期为8、6、4位整数:\verb|20101225|、\verb|101225|、\verb|1225| 分别代表三种不同的日期输入风格。
% \end{function}
% \begin{multicols}{2}
%     % \footnotesize
%     \begin{latexexample}
%   \begin{daily}[theorem,19111225]
%       那是一个遥远的圣诞夜。
%   \end{daily}
%   \end{latexexample}
%     \small
%     \begin{daily}[theorem,19111225]
%       那是一个遥远的圣诞夜。
%     \end{daily}
% \end{multicols}
% \begin{multicols}{2}
%     % \footnotesize
%     \begin{latexexample}
%   \begin{daily}[111225]
%       我忘不了那年的圣诞节。
%   \end{daily}
%   \end{latexexample}
%     \small
%     \begin{daily}[111225]
%         我忘不了那年的圣诞节。
%     \end{daily}
% \end{multicols}
% \begin{multicols}{2}
%     \begin{latexexample}
%   \begin{daily}[1225]
%       今年的圣诞节,还要多久?
%   \end{daily}
%   \end{latexexample}
%     \small
%     \begin{daily}[1225]
%         今年的圣诞节,还要多久?
%     \end{daily}
% \end{multicols}
% 日期未设置时,默认为当天日期 \verb|\today|:
% \begin{multicols}{2}
%     \begin{latexexample}
%   \begin{daily}
%       今天是\today{},不是么?
%   \end{daily}
%   \end{latexexample}
%     \small
%     \begin{daily}
%         今天是\today{},不是么?
%     \end{daily}
% \end{multicols}
% \begin{function}{难度}
%     \begin{fsyntax}
%     = (*<\defaultval{0}|1|2|3|4>*)
%     \end{fsyntax}
%   难度为0--4的整数,其产生的效果是在\pkg{tcolorbox}绘制的盒子左边绘制对应数量的星星。
% \end{function}
% \begin{daily}[theorem,4]
%     这是最高难度的题目。
% \end{daily}

% \begin{function}{标题}
% 不是可选难度的非数字参数均视为标题,这样的参数可能有多个,我们仅以第一个为准。
% \end{function}

% 以下是一个完整的例子:
% \begin{latexexample}
%   \begin{daily}[重要极限,不重要极限,20251112,1,lemma]
%   计算如下极限:
%   \[
%       \lim _ { x \to 0 } \frac { \sin x } { x }.
%   \]
%   \end{daily}
% \end{latexexample}
%   \begin{daily}[重要极限,不重要极限,20251112,1,lemma]
%   计算如下极限:
%   \[
%       \lim _ { x \to 0 } \frac { \sin x } { x }.
%   \]
%   \end{daily}


% \end{documentation}

% \EnableImplementation
% \begin{implementation}
% \newgeometry{
%   left      = 1 in,
%   right     = 1.00 in,
%   top       = 1.25 in,
%   bottom    = 1.00 in,
%   marginpar = 2.25 in
% }
% \section{实现细节}
%    \begin{macrocode}
%<@@=SYSUDaily>
%<*package>
%    \end{macrocode}
%    
% 首先是检查 \LaTeX3 环境.
%    \begin{macrocode}
\RequirePackage { xtemplate, l3keys2e }
\msg_new:nnn { SYSUDaily } { l3-too-old }
{
    Package~ `#1'~ is~ too~ old. \\\\
    Please~ update~ an~ up-to-date~ version~ of~ the~ bundles \\
    `l3kernel'~ and~ `l3packages'~ using~ your~ TeX~ package \\
    manager~ or~ from~ CTAN.
}
\clist_map_inline:nn { xtemplate, l3keys2e }
{
    \@ifpackagelater {#1} { 2020/07/17 }
    { } { \msg_error:nnn { SYSUDaily } { l3-too-old } {#1} }
}
%    \end{macrocode}
% \subsection{内部变量声明}
% 所有变量名中包含 |tmpa| / |tmpb| 的,均为临时变量。
% 变量名中包含 |low| 的,仅在盒子高度较低时起效,
% 变量名中包含 |high| 的,仅在盒子高度较高时起效。
%    \begin{macrocode}
\int_new:N \g_@@_counter_int
\int_new:N \l_@@_box_tmpa_difficulty_int
%    \end{macrocode}
% 在盒子高度较高时,\pkg{tcolorbox}盒子右端填充与默认一致,
% 盒子高度较低时,由于水印在盒子右边排版,此时盒子右端填充需要手动设置。
%    \begin{macrocode}
\dim_new:N \g_@@_box_right_low_dim
\dim_new:N \g_@@_box_right_high_dim
\dim_set:Nn \g_@@_box_right_high_dim { 2mm }
\dim_new:N \g_@@_box_indent_dim
\dim_new:N \g_@@_box_width_dim
\dim_new:N \g_@@_box_watermark_high_text_width_dim
\dim_new:N \g_@@_box_watermark_low_text_width_dim

\bool_new:N \c_@@_chinese_bool
\bool_new:N \g_@@_box_numbering_bool
\bool_new:N \g_@@_show_date_bool
\bool_new:N \g_@@_show_title_bool

\tl_new:N \g_@@_box_punctuation_after_title_tl
\tl_new:N \g_@@_box_after_title_tl
\tl_new:N \g_@@_box_watermark_high_text_tl
\tl_new:N \g_@@_box_watermark_low_text_tl
\tl_new:N \g_@@_box_before_watermark_high_text_tl
\tl_new:N \g_@@_box_before_watermark_low_text_tl
\tl_new:N \g_@@_box_after_watermark_high_text_tl
\tl_new:N \g_@@_box_after_watermark_low_text_tl
\tl_new:N \g_@@_box_watermark_high_tl
\tl_new:N \g_@@_box_watermark_low_tl
\tl_new:N \g_@@_difficulty_symbol_tl
\tl_new:N \l_@@_box_tmpa_title_tl
\tl_new:N \l_@@_box_tmpa_date_tl
%    \end{macrocode}
% 
% 为了颜色设置上的方便,我们让颜色混合值是一个标记列表而非浮点数,这样便可以通过
% 诸如 |!0!cyan| 的标记列表直接修改对应的颜色。
%    \begin{macrocode}
\tl_new:N \g_@@_box_background_color_mixing_tl
\tl_set:Nn \g_@@_box_background_color_mixing_tl { 3 }
\tl_new:N \g_@@_box_frame_color_mixing_tl
\tl_set:Nn \g_@@_box_frame_color_mixing_tl { 30 }
\tl_new:N \g_@@_box_tilte_color_mixing_tl
\tl_set:Nn \g_@@_box_tilte_color_mixing_tl { 100 }
\tl_new:N \g_@@_box_watermark_high_color_mixing_tl
\tl_set:Nn \g_@@_box_watermark_high_color_mixing_tl { 100 }
\tl_new:N \g_@@_box_watermark_low_color_mixing_tl
\tl_set:Nn \g_@@_box_watermark_low_color_mixing_tl { 100 }
\tl_new:N \g_@@_box_default_mathstyle_tl
\tl_new:N \g_@@_solution_name_tl
%    \end{macrocode}
% 
% 为了在排入\pkg{tcolorbox}盒子之前辨别高度,我们需要一个特殊的临时盒子。
%    \begin{macrocode}
\box_new:N \l_@@_tmpa_box
\box_new:N \l_@@_tmpa_measure_box

\seq_new:N \l_@@_tmpa_seq
\seq_new:N \c_@@_math_styles_seq
\seq_new:N \c_@@_math_styles_name_seq

\fp_new:N \l_@@_watermark_opacity_fp
\fp_set:Nn \l_@@_watermark_opacity_fp { 0.5 }
\fp_new:N \l_@@_watermark_low_opacity_fp
\fp_set:Nn \l_@@_watermark_low_opacity_fp { 0.5 }
\fp_new:N \l_@@_watermark_high_opacity_fp
\fp_set:Nn \l_@@_watermark_high_opacity_fp { 0.1 }
\fp_new:N \g_@@_watermark_low_xscale_fp
\fp_set:Nn \g_@@_watermark_low_xscale_fp { 1 }
\fp_new:N \g_@@_watermark_high_xscale_fp
\fp_set:Nn \g_@@_watermark_high_xscale_fp { 1 }
\fp_new:N \g_@@_watermark_low_yscale_fp
\fp_set:Nn \g_@@_watermark_low_yscale_fp { 1 }
\fp_new:N \g_@@_watermark_high_yscale_fp
\fp_set:Nn \g_@@_watermark_high_yscale_fp { 1 }

\cs_new:Nn \_@@_CJKecglue: { }
\cs_new:Nn \_@@_box_watermark_low_text_format: { }
\cs_new:Nn \_@@_box_watermark_high_text_format: { }
\cs_new:Nn \_@@_box_watermark_high_align: { }
\cs_new:Nn \_@@_box_watermark_low_align: { }
%    \end{macrocode}
% \subsection{宏包载入}
%    \begin{macrocode}
\RequirePackage [ skins, most, breakable ] { tcolorbox }
\RequirePackage { amsmath , amssymb , amsthm }
%    \end{macrocode}
% \subsection{键值设置}
%    \begin{macrocode}
\keys_define:nn { SYSUDaily } {
    language                  .choices:nn = { chinese, english } {
    \str_case:nn { #1 } {
            { chinese } { \bool_gset_true:N \c_@@_chinese_bool }
            { english } { \bool_gset_false:N \c_@@_chinese_bool }
        }
    },
    language                  .initial:n  = chinese,
    title / show              .bool_set:N = \g_@@_show_title_bool,
    title / show              .initial:n  = true,
    title / date              .bool_set:N = \g_@@_show_date_bool,
    title / date              .initial:n  = true,
    title / numbering         .bool_set:N = \g_@@_box_numbering_bool,
    title / numbering         .initial:n  = true,
    box / width               .dim_set:N  = \g_@@_box_width_dim,
    box / width               .initial:n  = 0.8 \textwidth,
    box / default-style       .tl_set:N   = \g_@@_box_default_mathstyle_tl,
    box / default-style       .initial:n  = proposition,
    title / punctuation       .tl_set:N   = \g_@@_box_punctuation_after_title_tl,
    title / after             .tl_set:N   = \g_@@_box_after_title_tl,
    box / indent              .dim_set:N  = \g_@@_box_indent_dim,
    box / low / padding-right .dim_set:N  = \g_@@_box_right_low_dim,
    background / symbol       .tl_set:N   = \g_@@_difficulty_symbol_tl,
    background / symbol       .initial:n  = { $ \bigstar $ },
    background / mixing       .tl_set:N   = \g_@@_box_background_color_mixing_tl,
    frame / mixing            .tl_set:N   = \g_@@_box_frame_color_mixing_tl,
    title / mixing            .tl_set:N   = \g_@@_box_tilte_color_mixing_tl,
    watermark / high / text   .tl_set:N   = \g_@@_box_watermark_high_text_tl,
    watermark / low / text    .tl_set:N   = \g_@@_box_watermark_low_text_tl,
    watermark / high / align  .choices:nn = { left, right, center } {
    \str_case:nn { #1 } {
            { left }    { \cs_set:Nn \_@@_box_watermark_high_align: { \raggedright } }
            { right }   { \cs_set:Nn \_@@_box_watermark_high_align: { \raggedleft } }
            { center }  { \cs_set:Nn \_@@_box_watermark_high_align: { \centering } }
        }
    },
    watermark / high / align  .initial:n  = center,
    watermark / low / align   .choices:nn = { left, right, center } {
    \str_case:nn { #1 } {
            { left }    { \cs_set:Nn \_@@_box_watermark_low_align: { \raggedright } }
            { right }   { \cs_set:Nn \_@@_box_watermark_low_align: { \raggedleft } }
            { center }  { \cs_set:Nn \_@@_box_watermark_low_align: { \centering } }
        }
    },
    watermark / low / align   .initial:n  = right,
    watermark / low / width   .dim_set:N  = \g_@@_box_watermark_low_text_width_dim,
    watermark / high / width  .dim_set:N  = \g_@@_box_watermark_high_text_width_dim,
    watermark / high / width  .dim_set:N  = \g_@@_box_watermark_high_text_width_dim,
    watermark / low / scale   .code:n     = {
            \fp_set:Nn \g_@@_watermark_low_xscale_fp { #1 }
            \fp_set:Nn \g_@@_watermark_low_yscale_fp { #1 }
        },
    watermark / high / scale  .code:n     = {
            \fp_set:Nn \g_@@_watermark_high_xscale_fp { #1 }
            \fp_set:Nn \g_@@_watermark_high_yscale_fp { #1 }
        },
    watermark / low / xscale  .dim_set:N  = \g_@@_watermark_low_xscale_fp,
    watermark / low / yscale  .dim_set:N  = \g_@@_watermark_low_yscale_fp,
    watermark / high / xscale .dim_set:N  = \g_@@_watermark_high_xscale_fp,
    watermark / high / yscale .dim_set:N  = \g_@@_watermark_high_yscale_fp,
    watermark / low / before  .tl_set:N   = \g_@@_box_before_watermark_low_text_tl,
    watermark / low / after   .tl_set:N   = \g_@@_box_after_watermark_low_text_tl,
    watermark / high / before .tl_set:N   = \g_@@_box_before_watermark_high_text_tl,
    watermark / high / after  .tl_set:N   = \g_@@_box_after_watermark_high_text_tl,
    watermark / low / format  .cs_set:Np  = \_@@_box_watermark_low_text_format:,
    watermark / high / format .cs_set:Np  = \_@@_box_watermark_high_text_format:,
    solution  / name          .tl_set:N   = \g_@@_solution_name_tl
}
\ProcessKeyOptions [ SYSUDaily ]
\NewDocumentCommand { \SYSUDailySetup } { m } {
    \keys_set:nn { SYSUDaily } { #1 }
}
%    \end{macrocode}
% \subsection{中英文设置}
%    \begin{macrocode}
\bool_if:NTF \c_@@_chinese_bool {
    \IfClassLoadedTF { ctexart } { } {
        \IfClassLoadedTF { ctexbook } { } {
            \IfClassLoadedTF { ctexbeamer } { } {
                \RequirePackage { ctex }
            }
        }
    }
    \sys_if_engine_uptex:T
    { \cs_set:Nn \_@@_CJKecglue: { \skip_horizontal:N \tex_xkanjiskip:D } }
    { \cs_set:Nn \_@@_CJKecglue: { \CJKecglue } }
    \seq_set_from_clist:Nn \c_@@_math_styles_name_seq {
        定理, 定义, 命题, 引理, 
        推论, 事实, 例子, 思考题, 
        计算题, 脑筋急转弯, { }
    }
    \dim_set:Nn \g_@@_box_watermark_low_text_width_dim { 4em }
    \dim_set:Nn \g_@@_box_watermark_high_text_width_dim { 150 pt }
    \dim_set:Nn \g_@@_box_right_low_dim { 5em + 2mm }
    \dim_set:Nn \g_@@_box_indent_dim { 2 \ccwd }
    \tl_set:Nn \g_@@_box_punctuation_after_title_tl { : }
    \tl_set:Nn \g_@@_box_after_title_tl { }
    \tl_set:Nn \g_@@_box_watermark_low_text_tl { 笃学笃行 \\ 不悔不倦 }
    \tl_set:Nn \g_@@_box_watermark_high_text_tl { 笃学笃行,不悔不倦。 }
    \tl_set:Nn \g_@@_box_before_watermark_high_text_tl { \tex_hss:D }
    \tl_set:Nn \g_@@_box_before_watermark_low_text_tl { \tex_hss:D
        \tex_vrule:D \@width 0.8pt
        \skip_horizontal:n { 1mm } }
    \tl_set:Nn \g_@@_box_after_watermark_low_text_tl { \skip_horizontal:n { 1.5 ex } }
    \tl_set:Nn \g_@@_solution_name_tl { 解 }
    \cs_set:Nn \_@@_box_watermark_low_text_format: { \bfseries }
    \cs_set:Nn \_@@_box_watermark_high_text_format: { \bfseries }
    \cs_set:Nn \_@@_set_date_format_aux:Nnnnnnnnn {
        \tl_set:Nn #1 {
            #2 #3 #4 #5
            \cs_if_eq:NNF #5 \scan_stop: {
                \l__zhnum_arabic_sep_tl \zhnum_output:n { year }
            }
            \l__zhnum_arabic_sep_tl
            \_@@_remove_zero_from_head:n { #6 } { #7 }
            \l__zhnum_arabic_sep_tl
            \zhnum_output:n { month }
            \l__zhnum_arabic_sep_tl
            \_@@_remove_zero_from_head:n { #8 } { #9 }
            \l__zhnum_arabic_sep_tl
            \zhnum_output:n { day }
        }
    }

} {
    \seq_set_from_clist:Nn \c_@@_math_styles_name_seq {
        Theorem, Definition, Proposition, Lemma, 
        Corollary, Fact, Example, Thought-provoking~ question, 
        Calculation~ question, Brain~ teaser, { }
    }
    \dim_set:Nn \g_@@_box_right_low_dim { 7em + 2mm }
    \dim_set:Nn \g_@@_box_indent_dim { 2 em }
    \tl_set:Nn \g_@@_box_punctuation_after_title_tl { : }
    \tl_set:Nn \g_@@_box_after_title_tl { \tex_space:D }
    \cs_set:Nn \_@@_box_watermark_low_text_format: { \linespread{1} \large \scshape }
    \cs_set:Nn \_@@_box_watermark_high_text_format: { \linespread{1} \Huge \bfseries }
    \tl_set:Nn \g_@@_box_watermark_low_text_tl {
        { \footnotesize T~h~e }
        \vbox_to_zero:n { \hbox_overlap_left:n { \color{ . ! 20 } \Huge \bfseries SYSU } } \\
        \skip_vertical:n { -0.4 ex }
        \color{ . ! 100 } School~ of \\
        Mathematics
    }
    \tl_set:Nn \g_@@_box_watermark_high_text_tl { Duxing }
    \tl_set:Nn \g_@@_box_before_watermark_low_text_tl { \tex_hss:D }
    \tl_set:Nn \g_@@_box_after_watermark_low_text_tl { 
        \skip_horizontal:n { 1mm } \color { . } 
        \tex_vrule:D \@width 0.8pt 
        \skip_horizontal:n { 3mm } 
    }
    \tl_set:Nn \g_@@_solution_name_tl { Solution }
    \dim_set:Nn \g_@@_box_watermark_low_text_width_dim { 100 pt }
    \dim_set:Nn \g_@@_box_watermark_high_text_width_dim { 150 pt }
    \fp_set:Nn \g_@@_watermark_low_xscale_fp { 0.75 }
    \fp_set:Nn \g_@@_watermark_low_yscale_fp { 0.75 }
    \cs_set:Nn \_@@_set_date_format_aux:Nnnnnnnnn {
        \tl_set:Nn #1 {
            \int_if_zero:nTF { #6 #7 } {
                \int_if_zero:nTF { #8 #9 } { \O } {
                    --- The~ \_@@_remove_zero_from_head:n { #8 } { #9 }
                    \int_case:nnF { #9 } {
                        { 1 } { \textsuperscript { st } }
                            { 2 } { \textsuperscript { nd } }
                            { 3 } { \textsuperscript { rd } }
                    } { \textsuperscript { th } }
                    \tex_space:D day
                },~ #2 #3 #4 #5
            } {
                \int_case:nnTF { #6 #7 } {
                    { 1 } { January }
                        { 2 } { February }
                        { 3 } { March }
                        { 4 } { April }
                        { 5 } { May }
                        { 6 } { June }
                        { 7 } { July }
                        { 8 } { August }
                        { 9 } { September }
                        { 10 } { October }
                        { 11 } { November }
                        { 12 } { December }
                } { } {
                    --- The~ \_@@_remove_zero_from_head:n { #6 } { #7 }
                    \int_case:nnF { #7 } {
                        { 1 } { \textsuperscript { st } }
                            { 2 } { \textsuperscript { nd } }
                            { 3 } { \textsuperscript { rd } }
                    } { \textsuperscript { th } }
                    \tex_space:D month,
                }
                \tex_space:D
                \int_if_zero:nTF { #8 #9 } { \O } { #8 #9 }
                ,~ #2 #3 #4 #5
            }
        }
    }
}

\int_step_inline:nn { \seq_count:N \c_@@_math_styles_seq } {
    \tl_set:cn { g_@@_box_ \seq_item:Nn \c_@@_math_styles_seq { #1 } _name_tl }
    { \seq_item:Nn \c_@@_math_styles_name_seq { #1 } }
}
%    \end{macrocode}  
%  
% \subsection{\pkg{SYSUDaily}环境}
% 定义如下十一个预设环境及对应颜色。
%    \begin{macrocode}
\seq_set_from_clist:Nn \c_@@_math_styles_seq {
    theorem, definition, proposition, lemma, 
    corollary, fact, example, think, calculate, 
    brain, empty
}
\definecolor { _@@_theorem_color } { RGB } { 12, 12, 150 }
\definecolor { _@@_definition_color } { RGB } { 220, 20, 60 }
\definecolor { _@@_proposition_color } { RGB } { 32, 115, 54 }
\definecolor { _@@_lemma_color } { RGB } { 0, 169, 153 }
\definecolor { _@@_corollary_color } { RGB } { 70, 130, 180 }
\definecolor { _@@_fact_color } { RGB } { 119, 136, 153 }
\definecolor { _@@_example_color } { RGB } { 255, 182, 193 }
\definecolor { _@@_think_color } { RGB } { 30, 144, 255 }
\definecolor { _@@_calculate_color } { RGB } { 139, 69, 19 }
\definecolor { _@@_brain_color } { RGB } { 0, 127, 127 }
\definecolor { _@@_empty_color } { RGB } { 0, 0, 0 }
%    \end{macrocode}
% 调整关于环境的键值设置。
%    \begin{macrocode}
\seq_map_inline:Nn \c_@@_math_styles_seq {
    \keys_define:nn { SYSUDaily } {
        daily / #1 / name .code:n = {
                \tl_set:cn { g_@@_box_ #1 _name_tl } { ##1 }
            },
        daily / #1 / color .code:n = {
                \colorlet { _@@_ #1 _color } { ##1 }
            }
    }
}
%    \end{macrocode}
% \subsection{\pkg{SYSUDaily}盒子}
% 首先利用\pkg{tcolorbox}设置盒子的共性。
%    \begin{macrocode}
\tcbset {
    mathstyles / .style = {
        before                 = { \medskip \tex_par:D \tex_noindent:D },
        after                  = { \skip_vertical:n { -3pt } },
        enhanced,
        toprule                = 3pt,
        fonttitle              = \bfseries,
        fontupper              = \slshape,
        arc                    = 0mm,
        outer~ arc             = 0mm,
        before~ skip~ balanced = 2mm,
        after~ skip~ balanced  = 3mm,
        boxrule                = 1pt,
        left                   = 3mm,
        top                    = 1mm,
        bottom                 = 1mm,
        lefttitle              = 0pt,
        righttitle             = 0pt,
        height~ from           = 6em~ to~ \paperheight,
        colback                = _@@_ #1 _color ! 
                                 \tl_use:N \g_@@_box_background_color_mixing_tl,
        colframe               = _@@_ #1 _color ! 
                                 \tl_use:N \g_@@_box_frame_color_mixing_tl,
        coltitle               = _@@_ #1 _color ! 
                                 \tl_use:N \g_@@_box_tilte_color_mixing_tl,
        watermark~ zoom        = 1,
    }
}
%    \end{macrocode}
%    
% 用来放置代表难度的星星的命令,未来可能会使用\pkg{l3draw}实现。
%    \begin{macrocode}
\cs_new:Nn \_@@_put_stars:nn {
    \begin{tcbclipinterior}
        \hbox_set:Nn \l_@@_tmpa_box {
            \footnotesize
            \color{ _@@_ #1 _color }
            \int_step_inline:nn { #2 } { \tl_use:N \g_@@_difficulty_symbol_tl }
        }
        \box_rotate:Nn \l_@@_tmpa_box { -90 }
        \path [ fill = none, draw = none ]
        ( interior . south~ west ) rectangle~ node [ white ]
            { \box_use:N \l_@@_tmpa_box }
        ( [ xshift = 4mm ] interior . north~ west );
    \end{tcbclipinterior}
}
%    \end{macrocode}
%    
% 用来排版盒子标题的命令。
%    \begin{macrocode}
\cs_new:Nn \_@@_box_parse_for_title:nnn {
    \bfseries
    \textsc { \use:c { g_@@_box_ #1 _name_tl } }
    \bool_if:NTF \c_@@_chinese_bool { } { ~ }
    \exp_args:Nf
    \IfBlankTF { #2 } { } {
        \bool_if:NT \g_@@_box_numbering_bool {
            \int_add:Nn \g_@@_counter_int { 1 }
            \int_use:N \g_@@_counter_int
        }
        \tl_use:N \g_@@_box_punctuation_after_title_tl
        \tl_use:N \g_@@_box_after_title_tl
        #2
    }
    \tex_hfill:D
    #3
}
%    \end{macrocode}
% 
% 当 |\g_@@_show_title_bool| 为 |false| 时,我们会借助禅盒子进行排版,此时需要保留除了标题外的一切。% 
%    \begin{macrocode}
\newtcolorbox { _@@_box } [ 8 ] {
    mathstyles         = #1,
    title              = \_@@_box_parse_for_title:nnn { #1 } { #4 } { #2 } 
                         \tex_kern:D -#6,
    watermark~ text    = { #5 },
    underlay           = { \_@@_put_stars:nn { #1 } { #3 } },
    watermark~ color   = _@@_ #1 _color ! #8,
    watermark~ opacity = #7,
    right              = { #6 }
}

\newtcolorbox { _@@_box_empty } [ 8 ] {
    mathstyles         = #1,
    watermark~ text    = { #5 },
    underlay           = { \_@@_put_stars:nn { #1 } { #3 } },
    watermark~ color   = _@@_ #1 _color ! #8,
    watermark~ opacity = #7,
    right              = { #6 }
}
%    \end{macrocode}
% 
% 分析填入的可选参数。
%    \begin{macrocode}
\cs_new:Nn \_@@_box_check:nnnnn {
    \seq_if_in:NnTF \c_@@_math_styles_seq { #1 } {
        \tl_set:Nn #2 { #1 }
    } {
%    \end{macrocode}
% 需要用正则来检测整数,然后才能判断大小。
%    \begin{macrocode}
        \regex_match:nnTF { ^\d+$ } { #1 } {
            \int_compare:nTF {
                #1 > 100
            } {
                \SYSUDaily_set_date_format:Nn #3 { #1 }
            } {
                \int_compare:nTF {
                    #1 < 5
                } {
                    \int_set:Nn #4 { #1 }
                } { }
            }
        } {
%    \end{macrocode}
% 并非属于 |\c_@@_math_styles_seq| 的非整数参数将视为标题。
%    \begin{macrocode}
            \tl_set:Nn #5 { #1 }
        }
    }
}
%    \end{macrocode}
% 调整日期的格式。
%    \begin{macrocode}
\cs_set:Npn \_@@_remove_zero_from_head:n #1 #2 {
    \int_compare:nTF { #1 = 0 } { #2 } { #1 #2 }
}
\cs_new:Nn \SYSUDaily_set_date_format:Nn {
    \bool_if:NT \g_@@_show_date_bool {
        \int_case:nnF { \tl_count:n { #2 } } {
            { 4 } { \_@@_set_date_format_aux:Nnnnnnnnn #1 
                    \scan_stop: \scan_stop: \scan_stop: \scan_stop: #2 }
            { 6 } { \_@@_set_date_format_aux:Nnnnnnnnn #1 
                        \scan_stop: \scan_stop: #2 }
            { 8 } { \_@@_set_date_format_aux:Nnnnnnnnn #1 #2 }
        } {
            \tl_set:Nn #1 { }
        }
    }
}
%    \end{macrocode}
% 最后排版。
%    \begin{macrocode}
\cs_set:Nn \_@@_box_high_aux:nnnnn {
    \str_if_eq:nnTF { #1 } { empty } {
        \begin { _@@_box_empty } { empty } { } { 0 } { } { }
        { 2mm } { } { 0 }
        #5
        \end { _@@_box_empty }
    } {
        \bool_if:NTF \g_@@_show_title_bool 
        { \begin { _@@_box } } { \begin { _@@_box_empty } }
        { #1 } { #2 } { #3 } { #4 }
        { \tl_use:N \g_@@_box_watermark_high_tl }
        { \dim_use:N \g_@@_box_right_high_dim }
        { \fp_use:N \l_@@_watermark_high_opacity_fp }
        { \tl_use:N \g_@@_box_watermark_high_color_mixing_tl }
        #5
        \bool_if:NTF \g_@@_show_title_bool 
        { \end { _@@_box } } { \end { _@@_box_empty } }
    }
}

\cs_set:Nn \_@@_box_low_aux:nnnnn {
    \str_if_eq:nnTF { #1 } { empty } {
        \begin { _@@_box_empty } { empty } { } { 0 } { } { }
        { 2mm } { } { 0 }
        #5
        \end { _@@_box_empty }
    } {
        \bool_if:NTF \g_@@_show_title_bool 
        { \begin { _@@_box } } { \begin { _@@_box_empty } }
        { #1 } { #2 } { #3 } { #4 }
        { \tl_use:N \g_@@_box_watermark_low_tl }
        { \dim_use:N \g_@@_box_right_low_dim }
        { \fp_use:N \l_@@_watermark_low_opacity_fp }
        { \tl_use:N \g_@@_box_watermark_low_color_mixing_tl }
        #5
        \bool_if:NTF \g_@@_show_title_bool 
        { \end { _@@_box } } { \end { _@@_box_empty } }
    }
}

\NewDocumentEnvironment { daily } { O { } +b } {
    \group_begin:
    \hbox_set:Nn \l_@@_tmpa_measure_box {
        \begin{minipage} { \g_@@_box_width_dim }
            \slshape #2
        \end{minipage}
    }
    \seq_set_from_clist:Nn \l_@@_tmpa_seq { #1 }
    \bool_if:NT \g_@@_show_date_bool {
        \tl_set:Ne \l_@@_box_tmpa_date_tl { \today }
    }
    \tl_set:Nn \l_@@_box_tmpa_mathstyle_tl 
    { \tl_use:N \g_@@_box_default_mathstyle_tl }
    \seq_reverse:N \l_@@_tmpa_seq
    \seq_map_inline:Nn \l_@@_tmpa_seq {
        \_@@_box_check:nnnnn { ##1 }
        { \l_@@_box_tmpa_mathstyle_tl }
        { \l_@@_box_tmpa_date_tl }
        { \l_@@_box_tmpa_difficulty_int }
        { \l_@@_box_tmpa_title_tl }
    }
%    \end{macrocode}
% 由于临时变量的原因,我们需要传值而非临时变量本身,因此在调用函数之前需要将其展开。
%    \begin{macrocode}
    \tex_expanded:D {
        \exp_args:Nf \dim_compare:nTF { \box_ht:N \l_@@_tmpa_measure_box > 26 pt }
        { \exp_not:N \_@@_box_high_aux:nnnnn }
        { \exp_not:N \_@@_box_low_aux:nnnnn }
        { \__kernel_exp_not:w \exp_after:wN { \exp:w \exp_end_continue_f:w 
          \tl_use:N \l_@@_box_tmpa_mathstyle_tl } }
        { \__kernel_exp_not:w \exp_after:wN { \exp:w \exp_end_continue_f:w 
          \tl_use:N \l_@@_box_tmpa_date_tl } }
        { \__kernel_exp_not:w \exp_after:wN { \exp:w \exp_end_continue_f:w 
          \int_use:N \l_@@_box_tmpa_difficulty_int } }
        { \__kernel_exp_not:w \exp_after:wN { \exp:w \exp_end_continue_f:w 
          \tl_use:N \l_@@_box_tmpa_title_tl } }
    } {
        \dim_set_eq:NN \tex_parindent:D \g_@@_box_indent_dim
        \tex_noindent:D
        #2
    }
    \group_end:
} { }
\NewDocumentEnvironment { solution } { } {
    \begin { proof } [ \tl_use:N \g_@@_solution_name_tl ]
        \slshape
} { \end { proof } }
%    \end{macrocode}
% \subsection{水印}
% 当盒子高度较低时,水印置右端排布,盒子高度较高时,水印置中心排布。
%    \begin{macrocode}
\tl_set:Nn \g_@@_box_watermark_low_tl {
    \_@@_box_watermark_low_text_format:
    \hbox_set:Nn \l_@@_tmpa_box {
        \tl_use:N \g_@@_box_before_watermark_low_text_tl
        \parbox
        { \g_@@_box_watermark_low_text_width_dim }
        { \_@@_box_watermark_low_align: 
        \tl_use:N \g_@@_box_watermark_low_text_tl }
        \tl_use:N \g_@@_box_after_watermark_low_text_tl
    }
    \box_scale:Nnn \l_@@_tmpa_box
    { \g_@@_watermark_low_xscale_fp  }
    { \g_@@_watermark_low_yscale_fp }
    \hbox_to_wd:nn { \textwidth } {
        \tex_hss:D
        \box_use:N \l_@@_tmpa_box
    }
}

\tl_set:Nn \g_@@_box_watermark_high_tl {
    \_@@_box_watermark_high_text_format:
    \hbox_set:Nn \l_@@_tmpa_box {
        \tl_use:N \g_@@_box_before_watermark_high_text_tl
        \parbox
        { \g_@@_box_watermark_high_text_width_dim }
        { \_@@_box_watermark_high_align: 
        \tl_use:N \g_@@_box_watermark_high_text_tl }
        \tl_use:N \g_@@_box_after_watermark_high_text_tl
    }
    \box_scale:Nnn \l_@@_tmpa_box
    { \g_@@_watermark_high_xscale_fp  }
    { \g_@@_watermark_high_yscale_fp }
    \box_use:N \l_@@_tmpa_box
}
%</package>
%<@@=>
%    \end{macrocode}
% \end{implementation}